
\begin{DoxyCodeInclude}
\textcolor{preprocessor}{# }
\textcolor{preprocessor}{}\textcolor{preprocessor}{#   Generic Makefile for simple projects.}
\textcolor{preprocessor}{}\textcolor{preprocessor}{#}
\textcolor{preprocessor}{}\textcolor{preprocessor}{#   (C) 2015, A.W. Janisse}
\textcolor{preprocessor}{}\textcolor{preprocessor}{#}
\textcolor{preprocessor}{}\textcolor{preprocessor}{#   Macro's:}
\textcolor{preprocessor}{}\textcolor{preprocessor}{#       OUTPUT  : Name of the executable}
\textcolor{preprocessor}{}\textcolor{preprocessor}{#       PIDIR   : Place for installation on the Raspberry PI}
\textcolor{preprocessor}{}\textcolor{preprocessor}{#       CC      : Default compiler. (Note make pi will build for the Raspberry Pi platform)}
\textcolor{preprocessor}{}\textcolor{preprocessor}{#       LIBS    : Libraries to use when building}
\textcolor{preprocessor}{}\textcolor{preprocessor}{#       CFLAGS  : Compiler flags}
\textcolor{preprocessor}{}\textcolor{preprocessor}{#       ZIPDIR  : Directory to put the backup archive files}
\textcolor{preprocessor}{}\textcolor{preprocessor}{#}
\textcolor{preprocessor}{}
OUTPUT  = xbc
PIDIR   = root@10.0.0.42:/bin
CC      = gcc
LIBS    = -lusb-1.0
CFLAGS  = -O2 -Wall -Werror
ZIPDIR  = ../backup

\textcolor{preprocessor}{### -----[ Do not change anything below this line ]----- ###}
\textcolor{preprocessor}{}
\textcolor{preprocessor}{# Remove any unwanted leading and trailing spaces}
\textcolor{preprocessor}{}TARGET = $(strip $(OUTPUT))
\textcolor{preprocessor}{# Retreive a list of source files (ending with .c)}
\textcolor{preprocessor}{}SOURCES = $(wildcard *.c)
# Replace all .c in the sources list to .o
OBJECTS = $(patsubst %.c, %.o, $(SOURCES))
# Retreive a list of header files (ending with .h)
HEADERS = $(wildcard *.h)   
# Build the archive name and set extension
TARNAME = $(ZIPDIR)/$(TARGET)\_$(shell date +\textcolor{stringliteral}{'%Y%m%d\_%H%M'})$(TAREXT)
TAREXT  = .tar

.PHONY: all debug clean install info html pdf backup pi

# Rule to perform when just make is executed.
all: $(TARGET)

\textcolor{preprocessor}{# implicit rule for building the object files.}
\textcolor{preprocessor}{}%.o: %.c $(HEADERS)
    $(CC) $(CFLAGS) -c $< -o $@

# Don\textcolor{stringliteral}{'t delete intermediate files when this make is aborted for some reason.}
\textcolor{stringliteral}{.PRECIOUS: $(TARGET) $(OBJECTS)}
\textcolor{stringliteral}{}
\textcolor{stringliteral}{# Here the compiling hapens }
\textcolor{stringliteral}{$(TARGET): $(OBJECTS)}
\textcolor{stringliteral}{    $(CC) $(OBJECTS) -Wall $(LIBS) -o $@}
\textcolor{stringliteral}{}
\textcolor{stringliteral}{# Build all with debug information. The X86 toolchain is used.}
\textcolor{stringliteral}{debug: CFLAGS += -g}
\textcolor{stringliteral}{debug: all}
\textcolor{stringliteral}{}
\textcolor{stringliteral}{# Build all with the arm-linux-gcc toolchain.}
\textcolor{stringliteral}{pi: CC=arm-linux-gcc}
\textcolor{stringliteral}{pi: all}
\textcolor{stringliteral}{}
\textcolor{stringliteral}{# Just cleanup by removing the exectable en .obj files.}
\textcolor{stringliteral}{clean:}
\textcolor{stringliteral}{    @rm -rf $(TARGET) $(OBJECTS)}
\textcolor{stringliteral}{}
\textcolor{stringliteral}{# Give information about this Makefile.}
\textcolor{stringliteral}{info: }
\textcolor{stringliteral}{    @echo ==================================================================}
\textcolor{stringliteral}{    @echo Output"   ":   $(TARGET)}
\textcolor{stringliteral}{    @echo Sources"  ": $(SOURCES)}
\textcolor{stringliteral}{    @echo Headers"  ": $(HEADERS)}
\textcolor{stringliteral}{    @echo Objects"  ": $(OBJECTS)}
\textcolor{stringliteral}{    @echo Libraries: $(LIBS)}
\textcolor{stringliteral}{    @echo Compiler : $(CC)}
\textcolor{stringliteral}{    @echo CFlags"   ": $(CFLAGS)}
\textcolor{stringliteral}{    @echo Zip dir"  ": $(ZIPDIR)}
\textcolor{stringliteral}{    @echo ==================================================================}
\textcolor{stringliteral}{}
\textcolor{stringliteral}{# Copy the executable over to the Raspberry PI.}
\textcolor{stringliteral}{install:}
\textcolor{stringliteral}{    @echo Connecting...}
\textcolor{stringliteral}{    @scp $(TARGET) $(PIDIR)}
\textcolor{stringliteral}{    @echo installation done!}
\textcolor{stringliteral}{}
\textcolor{stringliteral}{# Create an archive file containing all the essential files for reproduction.}
\textcolor{stringliteral}{backup: clean}
\textcolor{stringliteral}{backup:}
\textcolor{stringliteral}{    @mkdir -p $(ZIPDIR)}
\textcolor{stringliteral}{    @tar -cf $(TARNAME) .}
\textcolor{stringliteral}{    @echo Created archive: $(TARNAME)}
\textcolor{stringliteral}{}
\textcolor{stringliteral}{# Produce the HTML documentation based on the settings in Doxyfile.}
\textcolor{stringliteral}{html:}
\textcolor{stringliteral}{    @doxygen}
\textcolor{stringliteral}{    @echo HTML documentation is generated in doc/html}
\textcolor{stringliteral}{}
\textcolor{stringliteral}{# Produce the PDF documentation based on the settings in Doxyfile.}
\textcolor{stringliteral}{pdf:}
\textcolor{stringliteral}{    $(MAKE) -C doc/latex}
\textcolor{stringliteral}{    @echo PDF documentation is generated in doc/latex}
\end{DoxyCodeInclude}
 